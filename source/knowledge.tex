\section{知識問題}

\subsubsection{ポアソン過程}
\barquo{
ポアソン過程の定義を書け。

\rightline{\cite{教科書}p.8-13〜8-14}
}

\begin{sol}

クレーム件数過程$\{N_t\}$が次の条件を満たすとき、そのような過程をポアソン過程と呼ぶ。

\begin{enumerate}
  \item $0 \leq s < t \leq u < v$のとき$N_t - N_s$と$N_v - N_u$は独立。(加法過程)
  \item $N_{s+t} - N_t$と$N_s$は同じ分布に従う。(定常過程)
  \item $P(N_{t+h} - N_t \geq 2) = o(h) $
\end{enumerate}

また、同値な命題として、以下がある。

$\lambda = - \log P(N_1 = 0)$として、任意の$t$に対して、
\begin{equation}
  P(N_t = n) = \frac{(\lambda t)^n}{n!} e^{-\lambda t}
  \label{eq:ポアソン過程}
\end{equation}
が成立。式(\ref{eq:ポアソン過程})がポアソン分布に似ているので、ポアソン過程と呼ばれている。

\end{sol}

\

\newpage

\subsubsection{オペレーショナルタイム}
\barquo{
オペレーショナルタイムとは何か。


\rightline{\cite{教科書}p.8-16〜8-17}
}

\begin{sol}
$ $

ポアソン過程の定義の二つ目の条件(定常過程)は、実際問題への応用を考えたとき、大きな制約となる。
(何かが起こる確率は、どれだけ時間が経過したかだけではなく、季節や時期に依存するのが普通)

クレームの発生率を時期によらず安定させるために、時間尺度を伸縮させるための関数がオペレーショナルタイムである。

オペレーショナルタイム$\tau(t) $は、以下で定義される。

\[
  \tau(t) = - \log P(N_t = 0)
\]


\;

(メモ)

定常過程の条件を以下に入れ替えると、その確率過程は非斉時ポアソン過程に従う。
\[
  f_0(t) = P(N_t = 0)はtの関数として連続
\]

非斉時ポアソン過程では、
\[
  P(N_t = n) = \frac{\tau(t)^n}{n!} e^{-\tau(t)}
\]
となる。
ちなみに、定常過程のとき$\tau(t) = \lambda t$となり、これはポアソン過程の$P(N_t = n)$の式と一致する。

強度関数を$\displaystyle \lambda(t) = \frac{d\tau(t)}{dt}$と定義すると、
非斉時ポアソン過程は、ポアソン過程の$\lambda$を$\lambda(t)$に置き換えたものだと解釈できる。


\end{sol}

\

\newpage

\subsubsection{調整係数}
\barquo{

調整係数とは何か。

\rightline{\cite{教科書}}
}

\begin{sol}

$ $

$E(e^{-r(c-W_i)}) = 1$の正の解を調整係数と呼ぶ。
ただし、$c$は単位時間あたりの収入保険料であり、$W_i$は単位時間あたりのクレーム総額を表す確率変数である。


\end{sol}

\
\newpage


\subsubsection{ネット再保険料・グロス再保険料}
\barquo{
ネット再保険料とグロス再保険料の違いを簡記せよ。


\rightline{\cite{教科書}p.9-11}
}

\begin{sol}
$ $

\textbf{ネット再保険料}:再保険者の負担する保険金の期待値。

\textbf{グロス再保険料}:ネット再保険料に経費、手数料、利潤、安全割増等の付加保険料を加えたもの。

\end{sol}

\



\newpage

\subsubsection{定理9.2と定理9.3}
\barquo{

\begin{description}
  \item[(i)] ネット再保険料が等しい再保険処理の中で、保有保険金の分散を最小にするものを次の選択肢から選べ。
  \item[(ii)] 保有保険金の分散が等しい再保険処理の中で、再保険者の分散を最小にするものを次の選択肢から選べ。
\end{description}

$ $

(選択肢)

比例再保険、損害額再保険、超過損害額再保険、ストップロス再保険

\rightline{\cite{教科書}p.9-18〜p.9-19}
}

\begin{sol}
  \;

  \begin{description}
    \item[(i)] ストップロス再保険
    \item[(ii)] 比例再保険
  \end{description}
\end{sol}

\

\textbf{(メモ)}

(i)、(ii)ともにあらゆる再保険処理の中で分散最小となる。
証明のロジックだけ書く(詳細は\cite{教科書}参照)。

(定理9-1)保険料が同一な再保険処理なら、いかなる形の非関数型再保険より分散の小さい関数型再保険が存在。

(定理9-2)ストップロス再保険の分散と一般の関数型再保険の分散を比較。


\

ストップロス再保険は出再会社に有利。
比例再保険は受再会社に有利。


\newpage



\subsubsection{フレシェ・グンベル・ワイブル}
\barquo{

次の分布がどの最大値吸引域に属するか答えよ。

$ $

コーシー分布、パレート分布、$t$分布、フレシェ分布、
指数分布、ワイブル分布、ガンマ分布、正規分布、対数正規分布、グンベル分布、
一様分布、ベータ分布、ワイブル極値分布

\rightline{\cite{アク数学シリーズ}p.206}
}

\begin{sol}
$ $

対数正規分布以外は以下の基準で判別可能である。
\begin{itemize}
  \item ある次元以上のモーメントが存在しない$\iff$フレシェ型($F_X(x) = \exp\{-x^{-\alpha}\}, (x > 0)$ )
  \item 取りうる値の上限はないが、全ての次元のモーメントが存在する\\
  $\iff$グンベル型($F_X(x) = \exp\{-e^{-x}\},  (-\infty < x < \infty)$ ) 
  \item 取りうる値に上限がある$\iff$ワイブル分布($F_X(x) = \exp\{-(-x)^{\alpha}\}, (x < 0)$)
\end{itemize}

対数正規分布は例外的な事例であり、すべての次元にモーメントは存在しないがグンベル型の最大値吸引域に属する。
よって答えは、
\begin{itemize}
  \item フレシェ型:コーシー分布、パレート分布、$t$分布、フレシェ分布
  \item グンベル型:指数分布、ワイブル分布、ガンマ分布、正規分布、対数正規分布、グンベル分布
  \item ワイブル型:一様分布、ベータ分布、ワイブル極値分布
\end{itemize}

\end{sol}

\


\textbf{(補足)}

一般の確率変数$X$が一般化極値分布の最大値吸引域に属するための必要十分条件は
\cite{教科書}には記載がない。
アクチュアリー試験の範囲を超過するので、対策しなくてOK。たぶん。

\newpage

\subsubsection{構造的モデル・統計的モデル}
\barquo{
複数のリスクファクター同士の間で何らかの従属性が示唆されるとき、
その取り扱いについて、構造的モデルと統計的モデルという2通りの考え方がある。
それぞれの特徴を簡記せよ。

\rightline{\cite{教科書}10-21〜10-22}
}

\begin{sol}
$ $

風災害があれば火災も同時に起きやすいといった特定された因果関係に基づいて関連性を記述するモデルを\textbf{構造的モデル}
と呼ぶ。
一方、因果関係を特定できないときに統計的性質に基づいて関連性を表現するモデルを\textbf{統計的モデル}と呼ぶ。
\end{sol}


\


\textbf{(メモ)}

\cite{教科書}には主に統計的モデルによる従属性の取り扱い方が載っている。


\newpage

\subsubsection{相関係数}
\barquo{
ピアソンの相関係数($r_{xy} = \frac{Cov(x, y)}{ \sigma(x) \sigma(y)}$)
を従属性の指標として用いるには不適切な場合が存在する。
どのような場合か。

\rightline{\cite{教科書}10-25〜10-27}
}

\begin{sol}
$ $

\begin{itemize}
  \item 二つの変数の間に完全な従属性があっても、$r_{xy}$が大きくなるとは限らない。(\cite{教科書}例10.8)
  \item 全く同じ従属性であっても、$r_{xy}$の値は周辺分布によって異なる場合がある。(\cite{教科書}例10.9)
  \item $r_{xy}$の取りうる範囲は周辺分布およびそのパラメータに依存する。(\cite{教科書}例10.10)
\end{itemize}
\end{sol}

\


\textbf{(メモ)}

ケンドールの$\tau$やスペアマンの$\rho$は周辺分布に依存しない。


\newpage

\subsubsection{コピュラのパラメータ推定法}
\barquo{
コピュラのパラメータ推定法には、順位相関係数による積率法、最尤法、IFM法、規準最尤法がある。
それぞれについて簡記せよ。

\rightline{\cite{教科書}10-40〜10-41}
}

\begin{sol}
$ $

\textbf{<順位相関係数による積率法>}

コピュラによって計算されるケンドールの$\tau$が、
観測データによって求められるケンドールの$\tau$と一致するものとして、
コピュラのパラメータを推定する方法。
最尤法やIFM法と異なり、周辺分布を特定する必要がないというメリットがある。


\textbf{<最尤法>}

コピュラの密度関数から対数尤度関数を定め、それを最大化するパラメータを求める方法。
周辺分布のパラメータとコピュラのパラメータを同時に推定する方法であり、
コピュラの次元$N$が大きいと推定が困難になる。

\textbf{<IFM法>}

先に周辺分布のパラメータを推定し、それを元に最尤法を行いコピュラのパラメータを推定する方法。

\textbf{<規準最尤法>}

最尤法やIFM法には、周辺分布のパラメータの推定誤り影響をコピュラのパラメータの推定値が受けるという欠点がある。
それを回避するために、観測データを分位点データに変換し、最尤法によりパラメータを推定する方法が規準最尤法である。

\end{sol}

\newpage

\subsubsection{リスク尺度}
\barquo{
コヒーレント・リスク尺度、凸リスク尺度、歪みリスク尺度の定義を書け。

\rightline{\cite{教科書}10-48〜10-52}
}

\begin{sol}
$ $

\textbf{<コヒーレント・リスク尺度>}
\begin{enumerate}
  \item 平行移動不変性:任意の実数$c$に対し、$\rho(X+c) = \rho(X) + c$
  \item 単調性:$P(X \leq Y ) = 1$ならば$\rho(X) \leq \rho(Y)$
  \item 劣加法性:$\rho(X+Y) \leq \rho(X) + \rho(Y)$
  \item 正の同次性:任意の実数$c$に対し、$\rho(cX) = c\rho(X) $
\end{enumerate}

\textbf{<凸リスク尺度>}
\begin{enumerate}
  \item 平行移動不変性
  \item 単調性
  \item 凸性:$0 \leq c \leq 1$に対し、$\rho(c X + (1-c)Y) \leq c\rho(X) + (1-c) \rho(Y)$
\end{enumerate}

\textbf{<歪みリスク尺度>}
\begin{enumerate}
  \item 平行移動不変性
  \item 単調性
  \item 正の同次性
  \item 歪み関数$g$が凹関数の場合 $\iff$ 劣加法性 \\
  歪み関数$g$が凸関数の場合 $\iff$ 優加法性:$\rho(X+Y) \geq \rho(X) + \rho(Y)$
\end{enumerate}


\end{sol}

\


\textbf{(メモ)}

コヒーレント・リスク尺度ならば凸リスク尺度である(逆は成り立たない)。

