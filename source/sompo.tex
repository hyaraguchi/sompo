\documentclass[10pt]{jsarticle}%文字サイズが10ptのjsarticle

\input{preamble.tex}


\setcounter{tocdepth}{3}%目次に含めるレベル。1ならsectionまで。2ならsubsectionまで。3ならsubsubsectionまで。
\begin{document}



\title{損保数理の前提知識}
\author{hyaraguchi}
\date{2024年2月25日}
\maketitle




\tableofcontents%目次

\newpage

\section{はじめに}


本資料は公益社団法人日本アクチュアリー会の資格試験「損保数理」の対策問題集です。
損保数理WBでは演習量が足りなかったり、前提知識として扱われている分野の問題を中心に蒐集しております。
問題は、一部自作問題も含んでおりますが、基本的には日本アクチュアリー会HP(\url{https://www.actuaries.jp/lib/collection/})や参考文献リストの著作から引用しております。

本資料はhyaraguchiの私的利用を目的に作成されたものです。
著作権者に著作物の使用許可を取っておりませんので、本資料の複写・転載を禁じます。
個人の学習の範囲内でご使用ください。

解答の配布はしておりません。
どうしても欲しい方はhyaraguchi宛にご連絡ください。
本資料を持っている方はhyaraguchi周辺のごく限られた範囲内に限られているはずですので、連絡先は明記しません。
ご理解のほどよろしくお願いいたします。


\newpage

\section{数学 常識問題}

\subsubsection{ガウス積分}
\barquo{

以下のような定積分を\textgt{ガウス積分}と呼ぶ。
\begin{equation}
  \int_{-\infty}^{\infty} e^{-ax^2} \ dx = \sqrt{\frac{\pi}{a}}
  \label{eq:ガウス積分}
\end{equation}

\begin{description}
  \item[(i)] 式(\ref{eq:ガウス積分})を導出せよ。
  \item[(ii)] 
  式(\ref{eq:正規分布の密度関数})で定める関数$f(x)$を$x$の全区間で積分したとき、その値が$1$となることを確かめよ。
    \begin{equation}
      f(x) = \frac{1}{\sqrt{2 \pi} \sigma} \exp \left\{ - \frac{(x - \mu)^2}{2 \sigma^2} \right\}
      \label{eq:正規分布の密度関数}
    \end{equation}

  \;
  

\end{description}

(追記)

式(\ref{eq:正規分布の密度関数})は正規分布の確率密度関数である。
この関数はガウス積分の被積分関数を適当に平行移動させて、適当に定数倍したような形をしている。

\rightline{参考:\cite{数研微積分}p.268-269}
}

\newpage

\subsubsection{ガンマ関数}
\barquo{
任意の正の実数$s$に対して
\[
\Gamma (s) = \int_{0}^{\infty} e^{-x} x^{s-1} dx
\]
によって定める関数$\Gamma (s)$を\textgt{ガンマ関数}と呼ぶ。

ガンマ関数について以下の性質を証明せよ。
\begin{description}
  \item[(i)] 任意の正の実数$s$について、$\Gamma (s+1) = s\Gamma(s)$
  \item[(ii)] 任意の自然数$n$について、$\Gamma (n) = (n-1)!$
  \item[(iii)] $\displaystyle \Gamma \left ( \frac{1}{2} \right ) =  \sqrt{\pi}$
\end{description}
\rightline{参考:\cite{数研微積分}p.157, 270}
}

\newpage

\subsubsection{ベータ関数}
\barquo{
任意の正の実数$p, q$に対して
\[
B (p, q) = \int_{0}^{1} x^{p-1} (1-x)^{q-1} dx
\]
によって定める関数$B (p, q)$を\textgt{ベータ関数}と呼ぶ。

ベータ関数について以下の性質を証明せよ。
\begin{description}
  \item[(i)] 任意の正の実数$p, q$について、$B(p, q) = B(q, p)$
  \item[(ii)] 任意の正の実数$p, q$について、$\displaystyle B(p, q+1) = \frac{q}{p} B(p+1, q)$
  \item[(iii)] 任意の正の実数$p, q$について、$\displaystyle B(p, q) = \frac{\Gamma (p) \Gamma (q)}{\Gamma (p+q)}$
\end{description}
\rightline{参考:\cite{数研微積分}p.158}
}

\newpage

\subsubsection{テイラー展開・マクローリン展開}
\barquo{
関数$f(x)$が$x=a$の近傍で何回でも微分可能である(すなわち$C^{\infty}$級である)とき、
\[
f(x) = \sum_{n=0}^{\infty} \frac{f^{(n)} (a)}{n!}  (x-a)^n 
\]
を$f(x)$の$x=a$における\textgt{テイラー展開}と呼ぶ。
特に、$x=0$の場合
\[
f(x) = \sum_{n=0}^{\infty} \frac{f^{(n)} (0)}{n!}  x^n 
\]
を$f(x)$の\textgt{マクローリン展開}と呼ぶ。

次の関数のマクローリン展開を求めよ。
\begin{description}
  \item[(i)] $e^x$
  \item[(ii)] $\log (1+x)$
  \item[(iii)] $\displaystyle \frac{x}{(1-2x)^2}$ 
\end{description}
\rightline{参考:\cite{数研微積分}p.312}
}


\newpage

\subsubsection{ロピタルの定理}
\barquo{
$\displaystyle \lim_{x \rightarrow a} \frac{f(x)}{g(x)}$が$\displaystyle \frac{0}{0}$
または$\displaystyle \frac{\infty}{\infty}$
の不定形で「ある条件」を満たせば、
\[
  \lim_{x \rightarrow a} \frac{f(x)}{g(x)} =  \lim_{x \rightarrow a} \frac{f'(x)}{g'(x)}
\]
が成り立つ。
これを\textgt{ロピタルの定理}と呼ぶ。
(「ある条件」の詳細については参考文献に譲るが、損保数理で出てくる関数だったらだいたい条件を満たすはず。たぶん)

以下の問題に答えよ。
\begin{description}
  \item[(i)] $\displaystyle \lim_{x \rightarrow 0}\frac{\log (\cos x)}{x^2}$を求めよ。
  \item[(ii)] $\displaystyle \lim_{x \rightarrow 0}\frac{\log(x+1) - x \cos x}{1 - \cos x}$を求めよ。
  \item[(iii)] $\displaystyle \lim_{x \rightarrow \infty} x ^{1/x}$ を求めよ。
  \item[(iv)] 2次元クレイントンコピュラ$C(u, v) = (u^{-\alpha} + v^{-\alpha} - 1)^{- 1 / \alpha}$の右裾従属係数$\lambda_u$を求めよ。
\end{description}

\rightline{参考:\cite{数研微積分}p.112-113、\cite{美しい物語}\url{https://manabitimes.jp/math/748}}
}

\newpage

\newpage

\section{クレームの分析}

\subsubsection{混合分布}
\barquo{
確率変数$Y$の分布関数$F_Y (y)$が条件付分布関数$F_{X | \Theta} (y | \theta)$と密度関数$f_{\Theta} (\theta)$
を用いて、
\[
  F_Y (y) = \int_{-\infty}^{\infty} F_{X | \Theta} (y | \theta) f_{\Theta} (\theta) d\theta
\]
と表されるとき、$Y$の従う分布は、$F_{X | \Theta} (y | \theta)$によって表される分布の\textgt{混合分布}
であるという。

有名な混合分布の例を以下に列挙するので、これらが成り立つことを証明せよ。
\begin{description}
  \item[(i)] $N \sim Po(\Theta)$かつ$ \Theta \sim \Gamma(\alpha, \beta)$のとき、$N \sim NB \Big(\alpha, \frac{\beta}{1+\beta} \Big)$となる。
  \item[(ii)]  $N \sim Bin(\Theta, p)$かつ$ \Theta \sim Po(\lambda)$のとき、$N \sim Po(\lambda p)$となる。
  \item[(iii)] $N \sim N(\Theta, \sigma^2)$かつ$ \Theta \sim N(\mu_0, \sigma_0^2)$のとき、$N \sim  N(\mu_0, \sigma^2 + \sigma_0^2)$となる。
  \item[(iv)] $N \sim Bin(1, p)$かつ$\Theta \sim \beta eta(p, q)$のとき、$N \sim Bin \Big(1, \frac{p}{p+q} \Big)$
\end{description}

\rightline{参考:\cite{リスク・セオリー}p.66-68}
}

\newpage

\subsubsection{複合分布\textcircled{1}}
\barquo{
  $X_1, X_2, \cdots$を独立同一分布に従う確率変数の列とし、$N$を$X_1, X_2, \cdots$とは独立な確率変数とする。
  \[
    S = X_1 + X_2 + \cdots + X_N
  \]
  と$S$を定義するとき、$S$が従う分布は$X$の従う分布を$N$の従う分布で複合した\textgt{複合分布}であるという。

  有名な複合分布の例を以下に列挙するので、これらが成り立つことを証明せよ。
  \begin{description}
    \item[(i)] $X \sim Bin(1, p)$かつ$ N \sim Po(\lambda)$のとき、$S \sim Po(\lambda p)$となる。
    \item[(ii)]  $X \sim Bin(1, p)$かつ$ N \sim NB(\alpha, r)$のとき、$S \sim NB \Big(\alpha, \frac{r}{r + (1-r)p} \Big)$となる。
    \item[(iii)] $X \sim Bin(1, p)$かつ$ N \sim Bin(n, r)$のとき、$S \sim Bin (n, rp)$となる。
    \item[(iv)] $N \sim Po(\lambda)$かつ$N\sim LS(p)$(対数級数分布)のとき、$S \sim NB \Big(- \frac{\lambda}{\log p}, p \Big)$となる。
  \end{description}

  \rightline{参考:\cite{リスク・セオリー}p.69-72}
}

\newpage

\subsubsection{複合分布\textcircled{2}}
\barquo{
  複合分布の特性値を求めるための公式を以下に列挙する。
  これらが成り立つことを、条件付き期待値等の性質を使って証明せよ。
  \begin{description}
    \item[(i)] $E(S) = E(N)E(X)$
    \item[(ii)] $V(S) = E(N)V(X) + V(N)E(X)^2$
    \item[(iii)] $M_S(t) = M_N(\log M_X (t)) = M_N (C_X(t))$
  \end{description}

  \rightline{参考:\cite{リスク・セオリー}p.69-72}
}

\newpage



\newpage

\section{危険理論の基礎}

\subsubsection{Harrisの公式\textcircled{1}}
\barquo{
Lundbergモデルにおいて個々のクレーム額が指数分布$Exp(1/\mu)$に従うとき、破産確率$\varepsilon(u)$を求めよ。
ただし、安全割増率は$\theta$とする。

\rightline{\cite{難問題の系統}p.115、参考:\cite{アク数学シリーズ}p.100-101}
}

\newpage

\subsubsection{Harrisの公式\textcircled{2}}
\barquo{
Lundbergモデルにおいて個々のクレーム額が$Exp(3)$と$Exp(5)$の混合分布(混合比率1:2)に従うとき、破産確率$\varepsilon(u)$を求めよ。
ただし、安全割増率は$\theta = 4/11$とする。

\rightline{\cite{難問題の系統}p.115}
}

\newpage

\subsubsection{Harrisの公式\textcircled{3}}
\barquo{
Lundbergモデルにおいて個々のクレーム額が$\Gamma(2, 1/\beta)$に従うとき、破産確率$\varepsilon(u)$を求めよ。
ただし、安全割増率は$\theta = 7/8$とする。

\rightline{\cite{教科書}演習問題8.9 9}

}

\newpage


\newpage

\section{リスク評価の数理}

\subsubsection{コピュラ\textcircled{1}}
\barquo{
$X \sim U(0,1)$であり、ある$0 < p < 1$について
\begin{align*}
  Y = \begin{cases}
    \frac{p - X}{p} := g(X) & (0 \leq X \leq p) \\
    \frac{X - p}{1-p} := h(X)x& (p < X \leq 1)
  \end{cases}
\end{align*}
であるとき、$X$と$Y$に対するコピュラ$C(u, v)$を求めよ。

\rightline{\cite{アク数学シリーズ}例題11.1 p.232}
}

\newpage

\subsubsection{コピュラ\textcircled{2}}
\barquo{
$X$と$Y$に対するコピュラを$C(u, v)$とする。
$X$と$Z$が共単調であり、$Y$と$W$が共単調であるとき、
$Z$と$W$に対するコピュラを$C(u, v)$を用いて表せ。

\rightline{\cite{アク数学シリーズ}例題11.2 p.233}
}

\newpage

\subsubsection{コピュラ\textcircled{3}}
\barquo{
分布関数
\[
  F_{X, Y}(x, y) = \exp \Big(- (e^{- \theta x} + e^{ - \theta y})^{\frac{1}{\theta}} \Big)
\]
に対するコピュラがパラメータ$\theta$のグンベル・コピュラであることを示せ。

\rightline{\cite{アク数学シリーズ}例題11.3 p.240}
}

\newpage

\subsubsection{コピュラ\textcircled{4}}
\barquo{
確率変数ベクトル$(U, V)$のコピュラと$(1-U, 1-V)$のコピュラが同一であるとき、そのコピュラは\textgt{放射対称}であるという。

フランク・コピュラが放射対称であることを示せ。

\rightline{\cite{アク数学シリーズ}例題11.4 p.242}
}

\newpage

\subsubsection{コピュラ\textcircled{5}}
\barquo{
$X, Y$の同時密度関数が
\[
  f_{X, Y} (x, y) = \begin{cases}
    2 & (0 \leq x, y, x+y \leq 1) \\
    0 & (その他)
  \end{cases}
\]
であるとき、$X, Y$のケンドールの$\tau$とスピアマンの$\rho$を求めよ。

\rightline{\cite{アク数学シリーズ}例題11.5 p.266}
}

\newpage

\subsubsection{コピュラ\textcircled{6}}
\barquo{
$\phi(t)$が$0<t<1$で2回微分可能であるとき、
この$\phi$を生成作用素とするアルキメデス型コピュラに対する
ケンドールの$\tau$が
\[
\tau = 1 + 4 \int_{0}^{1} \frac{\phi(t)}{\phi'(t)} dt  
\]
であることを示せ。

\rightline{\cite{アク数学シリーズ}例題11.8 p.270}
}

\newpage

\newpage

\section{知識問題}

\subsubsection{ポアソン過程}
\barquo{
ポアソン過程の定義を書け。

\rightline{\cite{教科書}p.8-13〜8-14}
}

\begin{sol}

クレーム件数過程$\{N_t\}$が次の条件を満たすとき、そのような過程をポアソン過程と呼ぶ。

\begin{enumerate}
  \item $0 \leq s < t \leq u < v$のとき$N_t - N_s$と$N_v - N_u$は独立。(加法過程)
  \item $N_{s+t} - N_t$と$N_s$は同じ分布に従う。(定常過程)
  \item $P(N_{t+h} - N_t \geq 2) = o(h) $
\end{enumerate}

また、同値な命題として、以下がある。

$\lambda = - \log P(N_1 = 0)$として、任意の$t$に対して、
\begin{equation}
  P(N_t = n) = \frac{(\lambda t)^n}{n!} e^{-\lambda t}
  \label{eq:ポアソン過程}
\end{equation}
が成立。式(\ref{eq:ポアソン過程})がポアソン分布に似ているので、ポアソン過程と呼ばれている。

\end{sol}

\

\newpage

\subsubsection{オペレーショナルタイム}
\barquo{
オペレーショナルタイムとは何か。


\rightline{\cite{教科書}p.8-16〜8-17}
}

\begin{sol}
$ $

ポアソン過程の定義の二つ目の条件(定常過程)は、実際問題への応用を考えたとき、大きな制約となる。
(何かが起こる確率は、どれだけ時間が経過したかだけではなく、季節や時期に依存するのが普通)

クレームの発生率を時期によらず安定させるために、時間尺度を伸縮させるための関数がオペレーショナルタイムである。

オペレーショナルタイム$\tau(t) $は、以下で定義される。

\[
  \tau(t) = - \log P(N_t = 0)
\]


\;

(メモ)

定常過程の条件を以下に入れ替えると、その確率過程は非斉時ポアソン過程に従う。
\[
  f_0(t) = P(N_t = 0)はtの関数として連続
\]

非斉時ポアソン過程では、
\[
  P(N_t = n) = \frac{\tau(t)^n}{n!} e^{-\tau(t)}
\]
となる。
ちなみに、定常過程のとき$\tau(t) = \lambda t$となり、これはポアソン過程の$P(N_t = n)$の式と一致する。

強度関数を$\displaystyle \lambda(t) = \frac{d\tau(t)}{dt}$と定義すると、
非斉時ポアソン過程は、ポアソン過程の$\lambda$を$\lambda(t)$に置き換えたものだと解釈できる。


\end{sol}

\

\newpage

\subsubsection{調整係数}
\barquo{

調整係数とは何か。

\rightline{\cite{教科書}}
}

\begin{sol}

$ $

$E(e^{-r(c-W_i)}) = 1$の正の解を調整係数と呼ぶ。
ただし、$c$は単位時間あたりの収入保険料であり、$W_i$は単位時間あたりのクレーム総額を表す確率変数である。


\end{sol}

\
\newpage


\subsubsection{ネット再保険料・グロス再保険料}
\barquo{
ネット再保険料とグロス再保険料の違いを簡記せよ。


\rightline{\cite{教科書}p.9-11}
}

\begin{sol}
$ $

\textbf{ネット再保険料}:再保険者の負担する保険金の期待値。

\textbf{グロス再保険料}:ネット再保険料に経費、手数料、利潤、安全割増等の付加保険料を加えたもの。

\end{sol}

\



\newpage

\subsubsection{定理9.2と定理9.3}
\barquo{

\begin{description}
  \item[(i)] ネット再保険料が等しい再保険処理の中で、保有保険金の分散を最小にするものを次の選択肢から選べ。
  \item[(ii)] 保有保険金の分散が等しい再保険処理の中で、再保険者の分散を最小にするものを次の選択肢から選べ。
\end{description}

$ $

(選択肢)

比例再保険、損害額再保険、超過損害額再保険、ストップロス再保険

\rightline{\cite{教科書}p.9-18〜p.9-19}
}

\begin{sol}
  \;

  \begin{description}
    \item[(i)] ストップロス再保険
    \item[(ii)] 比例再保険
  \end{description}
\end{sol}

\

\textbf{(メモ)}

(i)、(ii)ともにあらゆる再保険処理の中で分散最小となる。
証明のロジックだけ書く(詳細は\cite{教科書}参照)。

(定理9-1)保険料が同一な再保険処理なら、いかなる形の非関数型再保険より分散の小さい関数型再保険が存在。

(定理9-2)ストップロス再保険の分散と一般の関数型再保険の分散を比較。


\

ストップロス再保険は出再会社に有利。
比例再保険は受再会社に有利。


\newpage



\subsubsection{フレシェ・グンベル・ワイブル}
\barquo{

次の分布がどの最大値吸引域に属するか答えよ。

$ $

コーシー分布、パレート分布、$t$分布、フレシェ分布、
指数分布、ワイブル分布、ガンマ分布、正規分布、対数正規分布、グンベル分布、
一様分布、ベータ分布、ワイブル極値分布

\rightline{\cite{アク数学シリーズ}p.206}
}

\begin{sol}
$ $

対数正規分布以外は以下の基準で判別可能である。
\begin{itemize}
  \item ある次元以上のモーメントが存在しない$\iff$フレシェ型($F_X(x) = \exp\{-x^{-\alpha}\}, (x > 0)$ )
  \item 取りうる値の上限はないが、全ての次元のモーメントが存在する\\
  $\iff$グンベル型($F_X(x) = \exp\{-e^{-x}\},  (-\infty < x < \infty)$ ) 
  \item 取りうる値に上限がある$\iff$ワイブル分布($F_X(x) = \exp\{-(-x)^{\alpha}\}, (x < 0)$)
\end{itemize}

対数正規分布は例外的な事例であり、すべての次元にモーメントは存在しないがグンベル型の最大値吸引域に属する。
よって答えは、
\begin{itemize}
  \item フレシェ型:コーシー分布、パレート分布、$t$分布、フレシェ分布
  \item グンベル型:指数分布、ワイブル分布、ガンマ分布、正規分布、対数正規分布、グンベル分布
  \item ワイブル型:一様分布、ベータ分布、ワイブル極値分布
\end{itemize}

\end{sol}

\


\textbf{(補足)}

一般の確率変数$X$が一般化極値分布の最大値吸引域に属するための必要十分条件は
\cite{教科書}には記載がない。
アクチュアリー試験の範囲を超過するので、対策しなくてOK。たぶん。

\newpage

\subsubsection{構造的モデル・統計的モデル}
\barquo{
複数のリスクファクター同士の間で何らかの従属性が示唆されるとき、
その取り扱いについて、構造的モデルと統計的モデルという2通りの考え方がある。
それぞれの特徴を簡記せよ。

\rightline{\cite{教科書}10-21〜10-22}
}

\begin{sol}
$ $

風災害があれば火災も同時に起きやすいといった特定された因果関係に基づいて関連性を記述するモデルを\textbf{構造的モデル}
と呼ぶ。
一方、因果関係を特定できないときに統計的性質に基づいて関連性を表現するモデルを\textbf{統計的モデル}と呼ぶ。
\end{sol}


\


\textbf{(メモ)}

\cite{教科書}には主に統計的モデルによる従属性の取り扱い方が載っている。


\newpage

\subsubsection{相関係数}
\barquo{
ピアソンの相関係数($r_{xy} = \frac{Cov(x, y)}{ \sigma(x) \sigma(y)}$)
を従属性の指標として用いるには不適切な場合が存在する。
どのような場合か。

\rightline{\cite{教科書}10-25〜10-27}
}

\begin{sol}
$ $

\begin{itemize}
  \item 二つの変数の間に完全な従属性があっても、$r_{xy}$が大きくなるとは限らない。(\cite{教科書}例10.8)
  \item 全く同じ従属性であっても、$r_{xy}$の値は周辺分布によって異なる場合がある。(\cite{教科書}例10.9)
  \item $r_{xy}$の取りうる範囲は周辺分布およびそのパラメータに依存する。(\cite{教科書}例10.10)
\end{itemize}
\end{sol}

\


\textbf{(メモ)}

ケンドールの$\tau$やスペアマンの$\rho$は周辺分布に依存しない。


\newpage

\subsubsection{コピュラのパラメータ推定法}
\barquo{
コピュラのパラメータ推定法には、順位相関係数による積率法、最尤法、IFM法、規準最尤法がある。
それぞれについて簡記せよ。

\rightline{\cite{教科書}10-40〜10-41}
}

\begin{sol}
$ $

\textbf{<順位相関係数による積率法>}

コピュラによって計算されるケンドールの$\tau$が、
観測データによって求められるケンドールの$\tau$と一致するものとして、
コピュラのパラメータを推定する方法。
最尤法やIFM法と異なり、周辺分布を特定する必要がないというメリットがある。


\textbf{<最尤法>}

コピュラの密度関数から対数尤度関数を定め、それを最大化するパラメータを求める方法。
周辺分布のパラメータとコピュラのパラメータを同時に推定する方法であり、
コピュラの次元$N$が大きいと推定が困難になる。

\textbf{<IFM法>}

先に周辺分布のパラメータを推定し、それを元に最尤法を行いコピュラのパラメータを推定する方法。

\textbf{<規準最尤法>}

最尤法やIFM法には、周辺分布のパラメータの推定誤り影響をコピュラのパラメータの推定値が受けるという欠点がある。
それを回避するために、観測データを分位点データに変換し、最尤法によりパラメータを推定する方法が規準最尤法である。

\end{sol}

\newpage

\subsubsection{リスク尺度}
\barquo{
コヒーレント・リスク尺度、凸リスク尺度、歪みリスク尺度の定義を書け。

\rightline{\cite{教科書}10-48〜10-52}
}

\begin{sol}
$ $

\textbf{<コヒーレント・リスク尺度>}
\begin{enumerate}
  \item 平行移動不変性:任意の実数$c$に対し、$\rho(X+c) = \rho(X) + c$
  \item 単調性:$P(X \leq Y ) = 1$ならば$\rho(X) \leq \rho(Y)$
  \item 劣加法性:$\rho(X+Y) \leq \rho(X) + \rho(Y)$
  \item 正の同次性:任意の実数$c$に対し、$\rho(cX) = c\rho(X) $
\end{enumerate}

\textbf{<凸リスク尺度>}
\begin{enumerate}
  \item 平行移動不変性
  \item 単調性
  \item 凸性:$0 \leq c \leq 1$に対し、$\rho(c X + (1-c)Y) \leq c\rho(X) + (1-c) \rho(Y)$
\end{enumerate}

\textbf{<歪みリスク尺度>}
\begin{enumerate}
  \item 平行移動不変性
  \item 単調性
  \item 正の同次性
  \item 歪み関数$g$が凹関数の場合 $\iff$ 劣加法性 \\
  歪み関数$g$が凸関数の場合 $\iff$ 優加法性:$\rho(X+Y) \geq \rho(X) + \rho(Y)$
\end{enumerate}


\end{sol}

\


\textbf{(メモ)}

コヒーレント・リスク尺度ならば凸リスク尺度である(逆は成り立たない)。






\newpage

\begin{thebibliography}{1}%参考文献の リスト
  \bibitem[過去問]{過去問} 公益社団法人 日本アクチュアリー会 資格試験過去問題集 \url{https://www.actuaries.jp/lib/collection/} (最終閲覧日:2023/12/10)
  \bibitem[教科書]{教科書} 日本アクチュアリー会『損保数理』(日本アクチュアリー会, 2011)
  \bibitem[モデリング]{モデリング} 日本アクチュアリー会『モデリング』(日本アクチュアリー会, 2005)
  \bibitem[リスク・セオリー]{リスク・セオリー} 岩沢宏和『リスク・セオリーの基礎』(培風館, 2010)
  \bibitem[アク数学シリーズ]{アク数学シリーズ} 岩沢宏和, 黒田耕嗣『アクチュアリー数学シリーズ4 損害保険数理』(日本評論社, 2015)
  \bibitem[ストラテジー]{ストラテジー} MAH, 平井卓也, 玉岡一史『アクチュアリー試験 合格へのストラテジー 損保数理』(東京図書, 2019)
  \bibitem[例題で学ぶ]{例題で学ぶ} 小暮雅一, 東出純『例題で学ぶ損害保険数理 第2版』(共立出版, 2016)
  \bibitem[難問題の系統]{難問題の系統}CAR他「難問題の系統とその解き方 損保数理」\url{}(最終閲覧日:2023/12/10)
  \bibitem[弱点克服]{弱点克服} 藤田岳彦『弱点克服 大学生の確率・統計』(東京図書, 2010)
  \bibitem[数研微積分]{数研微積分} 加藤文元『大学教養 微分積分』(数研出版, 2019)
  \bibitem[美しい物語]{美しい物語} マスオ「高校数学の美しい物語」\url{https://manabitimes.jp/math}(最終閲覧日:2023/12/10)
\end{thebibliography}


\end{document}
