\section{リスク評価の数理}

\subsubsection{コピュラ\textcircled{1}}
\barquo{
$X \sim U(0,1)$であり、ある$0 < p < 1$について
\begin{align*}
  Y = \begin{cases}
    \frac{p - X}{p} := g(X) & (0 \leq X \leq p) \\
    \frac{X - p}{1-p} := h(X)x& (p < X \leq 1)
  \end{cases}
\end{align*}
であるとき、$X$と$Y$に対するコピュラ$C(u, v)$を求めよ。

\rightline{\cite{アク数学シリーズ}例題11.1 p.232}
}

\newpage

\subsubsection{コピュラ\textcircled{2}}
\barquo{
$X$と$Y$に対するコピュラを$C(u, v)$とする。
$X$と$Z$が共単調であり、$Y$と$W$が共単調であるとき、
$Z$と$W$に対するコピュラを$C(u, v)$を用いて表せ。

\rightline{\cite{アク数学シリーズ}例題11.2 p.233}
}

\newpage

\subsubsection{コピュラ\textcircled{3}}
\barquo{
分布関数
\[
  F_{X, Y}(x, y) = \exp \Big(- (e^{- \theta x} + e^{ - \theta y})^{\frac{1}{\theta}} \Big)
\]
に対するコピュラがパラメータ$\theta$のグンベル・コピュラであることを示せ。

\rightline{\cite{アク数学シリーズ}例題11.3 p.240}
}

\newpage

\subsubsection{コピュラ\textcircled{4}}
\barquo{
確率変数ベクトル$(U, V)$のコピュラと$(1-U, 1-V)$のコピュラが同一であるとき、そのコピュラは\textgt{放射対称}であるという。

フランク・コピュラが放射対称であることを示せ。

\rightline{\cite{アク数学シリーズ}例題11.4 p.242}
}

\newpage

\subsubsection{コピュラ\textcircled{5}}
\barquo{
$X, Y$の同時密度関数が
\[
  f_{X, Y} (x, y) = \begin{cases}
    2 & (0 \leq x, y, x+y \leq 1) \\
    0 & (その他)
  \end{cases}
\]
であるとき、$X, Y$のケンドールの$\tau$とスピアマンの$\rho$を求めよ。

\rightline{\cite{アク数学シリーズ}例題11.5 p.266}
}

\newpage

\subsubsection{コピュラ\textcircled{6}}
\barquo{
$\phi(t)$が$0<t<1$で2回微分可能であるとき、
この$\phi$を生成作用素とするアルキメデス型コピュラに対する
ケンドールの$\tau$が
\[
\tau = 1 + 4 \int_{0}^{1} \frac{\phi(t)}{\phi'(t)} dt  
\]
であることを示せ。

\rightline{\cite{アク数学シリーズ}例題11.8 p.270}
}

\newpage