\section{危険理論の基礎}

\subsubsection{Harrisの公式\textcircled{1}}
\barquo{
Lundbergモデルにおいて個々のクレーム額が指数分布$Exp(1/\mu)$に従うとき、破産確率$\varepsilon(u)$を求めよ。
ただし、安全割増率は$\theta$とする。

\rightline{\cite{難問題の系統}p.115、参考:\cite{アク数学シリーズ}p.100-101}
}

\newpage

\subsubsection{Harrisの公式\textcircled{2}}
\barquo{
Lundbergモデルにおいて個々のクレーム額が$Exp(3)$と$Exp(5)$の混合分布(混合比率1:2)に従うとき、破産確率$\varepsilon(u)$を求めよ。
ただし、安全割増率は$\theta = 4/11$とする。

\rightline{\cite{難問題の系統}p.115}
}

\newpage

\subsubsection{Harrisの公式\textcircled{3}}
\barquo{
Lundbergモデルにおいて個々のクレーム額が$\Gamma(2, 1/\beta)$に従うとき、破産確率$\varepsilon(u)$を求めよ。
ただし、安全割増率は$\theta = 7/8$とする。

\rightline{\cite{教科書}演習問題8.9 9}

}

\newpage